
\documentclass[10pt,letterpaper]{article}
\usepackage[top=0.85in,left=2.75in,footskip=0.75in]{geometry}

% amsmath and amssymb packages, useful for mathematical formulas and symbols
\usepackage{amsmath,amssymb}

% Use adjustwidth environment to exceed column width (see example table in text)
\usepackage{changepage}

% textcomp package and marvosym package for additional characters
\usepackage{textcomp,marvosym}

% cite package, to clean up citations in the main text. Do not remove.
\usepackage{cite}

% Use nameref to cite supporting information files (see Supporting Information section for more info)
\usepackage{nameref,hyperref}

% line numbers
\usepackage[right]{lineno}

% ligatures disabled
\usepackage[nopatch=eqnum]{microtype}
\DisableLigatures[f]{encoding = *, family = * }

% color can be used to apply background shading to table cells only
%\usepackage[table]{xcolor}

% array package and thick rules for tables
\usepackage{array}

% create "+" rule type for thick vertical lines
\newcolumntype{+}{!{\vrule width 2pt}}

% create \thickcline for thick horizontal lines of variable length
\newlength\savedwidth
\newcommand\thickcline[1]{%
  \noalign{\global\savedwidth\arrayrulewidth\global\arrayrulewidth 2pt}%
  \cline{#1}%
  \noalign{\vskip\arrayrulewidth}%
  \noalign{\global\arrayrulewidth\savedwidth}%
}

% \thickhline command for thick horizontal lines that span the table
\newcommand\thickhline{\noalign{\global\savedwidth\arrayrulewidth\global\arrayrulewidth 2pt}%
\hline
\noalign{\global\arrayrulewidth\savedwidth}}


% Remove comment for double spacing
%\usepackage{setspace} 
%\doublespacing

% Text layout
\raggedright
\setlength{\parindent}{0.5cm}
\textwidth 5.25in 
\textheight 8.75in

% Bold the 'Fig #' in the caption and separate it from the title/caption with a period
% Captions will be left justified
\usepackage[aboveskip=1pt,labelfont=bf,labelsep=period,justification=raggedright,singlelinecheck=off]{caption}
\renewcommand{\figurename}{Fig}

% Use the PLoS provided BiBTeX style
\bibliographystyle{plos2015}

% Remove brackets from numbering in List of References
\makeatletter
\renewcommand{\@biblabel}[1]{\quad#1.}
\makeatother



% Header and Footer with logo
\usepackage{lastpage,fancyhdr,graphicx}
\usepackage{epstopdf}
\usepackage{lmodern}
%\pagestyle{myheadings}
\pagestyle{fancy}
\fancyhf{}
%\setlength{\headheight}{27.023pt}
%\lhead{\includegraphics[width=2.0in]{PLOS-submission.eps}}
\rfoot{\thepage/\pageref{LastPage}}
\renewcommand{\headrulewidth}{0pt}
\renewcommand{\footrule}{\hrule height 2pt \vspace{2mm}}
\fancyheadoffset[L]{2.25in}
\fancyfootoffset[L]{2.25in}
\lfoot{\today}

%% Include all macros below

\newcommand{\lorem}{{\bf LOREM}}
\newcommand{\ipsum}{{\bf IPSUM}}

%% END MACROS SECTION

%% personal packages and macro
%%% packages
\usepackage[utf8]{inputenc}        % allow utf-8 input
\usepackage[T1]{fontenc}           % use 8-bit T1 fonts
\usepackage[dvipsnames, table]{xcolor}
\usepackage{tabularx}
\usepackage{multirow}
\usepackage{pifont}
\usepackage{csvsimple}
\usepackage[font={small},textfont={it},labelfont={bf}]{caption}
\usepackage{subcaption}
\usepackage{graphicx}
\usepackage{url}                   % simple URL typesetting
\usepackage{booktabs}              % professional-quality tables
\usepackage{makecell}
\usepackage{amsfonts}              % blackboard math symbols
\usepackage{amsmath}
\usepackage{nicefrac}              % compact symbols for 1/2, etc.
\usepackage{microtype}             % microtypography
\usepackage{enumitem}
\usepackage[export]{adjustbox}


%not compatible with cite package
%\usepackage[natbib=true,style=nature,maxnames=999,maxcitenames=2,backend=biber]{biblatex}
%\addbibresource{references.bib}

%%% macros
\DeclareMathOperator*{\argmin}{arg\,min}
\newcommand{\indep}{\perp \!\!\! \perp}
\newtheorem{assumption}{Assumption}

\definecolor{dark_blue}{rgb}{0,0,.65}
\definecolor{dark_green}{rgb}{0,.5,.15}

\hypersetup{pdftex,  % needed for pdflatex
  breaklinks=true,  % so long urls are correctly broken across lines
  colorlinks=true,
  linkcolor=dark_blue,
  citecolor=dark_green,
}
\colorlet{P}{ForestGreen}
\colorlet{I}{MidnightBlue}
\colorlet{C}{YellowOrange}
\colorlet{O}{DarkOrchid}
\colorlet{T}{Gray}



\begin{document}
\vspace*{0.2in}

\section*{Supporting information}


\paragraph*{S4 Appendix}
\label{apd:packages}
{\bf Packages for causal estimation in the python ecosystem.}
We searched for causal inference packages in the python ecosystem. The focus
was on the identification methods. Important features were ease of
installation, sklearn estimator support, sklearn pipeline support, doubly
robust estimators, confidence interva l computation, honest splitting
(cross-validation), Targeted Maximum Likelihood Estimation. These criteria are
summarized in \ref{apd:table:causal_packages}. We finally chose EconML despite
lacking \texttt{sklearn.\_BaseImputer} support through the
\texttt{sklearn.Pipeline} object as well as a TMLE implementation.

The zEpid package is primarily intended for epidemiologists. It is well documented
and provides pedagogical tutorials. It does not support sklearn estimators,
pipelines and honest splitting.

EconML \cite{battocchi2019econml} implements almost all estimators except propensity score methods. Despite
focusing on Conditional Average Treatment Effect, it provides all. One downside
is the lack of support for scikit-learn pipelines with missing value imputers.
This opens the door to information leakage when imputing data before splitting
into train/test folds.

Dowhy \cite{sharma2020dowhy} focuses on graphical models and relies on EconML for most of the causal
inference methods (identifications) and estimators. Despite, being interesting
for complex inference --such as mediation analysis or instrumental variables--,
we considered that it added an unnecessary layer of complexity for our use case
where a backdoor criterion is the most standard adjustment methodology.

Causalml implements all methods, but has a lot of package dependencies
which makes it hard to install.

% Please add the following required packages to your document preamble:
% \usepackage{graphicx}
\begin{table}[h!]
  \resizebox{\textwidth}{!}{%
    \begin{tabular}{|l|l|l|l|l|l|l|l|l|}
      \hline
      \multicolumn{1}{|c|}{\textbf{Packages}}                                                           &
      \multicolumn{1}{c|}{\textbf{\begin{tabular}[c]{@{}c@{}}Simple \\ installation\end{tabular}}}      &
      \multicolumn{1}{c|}{\textbf{\begin{tabular}[c]{@{}c@{}}Confidence \\ Intervals\end{tabular}}}     &
      \textbf{\begin{tabular}[c]{@{}l@{}}sklearn \\ estimator\end{tabular}}                             &
      \textbf{\begin{tabular}[c]{@{}l@{}}sklearn \\ pipeline\end{tabular}}                              &
      \multicolumn{1}{c|}{\textbf{\begin{tabular}[c]{@{}c@{}}Propensity \\ estimators\end{tabular}}}    &
      \multicolumn{1}{c|}{\textbf{\begin{tabular}[c]{@{}c@{}}Doubly Robust \\ estimators\end{tabular}}} &
      \multicolumn{1}{c|}{\textbf{\begin{tabular}[c]{@{}c@{}}TMLE\\ estimator\end{tabular}}}            &
      \multicolumn{1}{c|}{\textbf{\begin{tabular}[c]{@{}c@{}}Honest splitting \\ (cross validation)\end{tabular}}} \\ \hline
      \textbf{\href{https://github.com/py-why/dowhy}{dowhy}}                                            &
      \ding{51}                                                                                         &
      \ding{51}                                                                                         &
      \ding{51}                                                                                         &
      \ding{51}                                                                                         &
      \ding{51}                                                                                         &
      \ding{55}                                                                                         &
      \ding{55}                                                                                         &
      \ding{55}                                                                                                    \\ \hline
      \textbf{\href{https://github.com/py-why/EconML}{EconML}}                                          &
      \ding{51}                                                                                         &
      \ding{51}                                                                                         &
      \ding{51}                                                                                         &
      \begin{tabular}[c]{@{}l@{}}Yes except\\ for imputers\end{tabular}                                 &
      \ding{55}                                                                                         &
      \ding{51}                                                                                         &
      \ding{55}                                                                                         &
      \begin{tabular}[c]{@{}l@{}}Only for doubly \\ robust estimators\end{tabular}                                 \\ \hline
      \textbf{\href{https://github.com/pzivich/zEpid}{zEpid}}                                           &
      \ding{51}                                                                                         &
      \ding{51}                                                                                         &
      \ding{55}                                                                                         &
      \ding{55}                                                                                         &
      \ding{51}                                                                                         &
      \ding{51}                                                                                         &
      \ding{51}                                                                                         &
      Only for TMLE                                                                                                \\ \hline
      \textbf{\href{https://github.com/uber/causalml}{causalml}}                                        &
      \ding{55}                                                                                         &
      \ding{51}                                                                                         &
      \ding{51}                                                                                         &
      \ding{51}                                                                                         &
      \ding{51}                                                                                         &
      \ding{51}                                                                                         &
      \ding{51}                                                                                         &
      \begin{tabular}[c]{@{}l@{}}Only for doubly \\ robust estimators\end{tabular}                                 \\ \hline
    \end{tabular}%
  }\caption{\textbf{Selection criteria for causal python packages.}}\label{apd:table:causal_packages}
\end{table}




\bibliography{references}


\end{document}
