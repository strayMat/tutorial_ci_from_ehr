
\documentclass[10pt,letterpaper]{article}
\usepackage[top=0.85in,left=2.75in,footskip=0.75in]{geometry}

% amsmath and amssymb packages, useful for mathematical formulas and symbols
\usepackage{amsmath,amssymb}

% Use adjustwidth environment to exceed column width (see example table in text)
\usepackage{changepage}

% textcomp package and marvosym package for additional characters
\usepackage{textcomp,marvosym}

% cite package, to clean up citations in the main text. Do not remove.
\usepackage{cite}

% Use nameref to cite supporting information files (see Supporting Information section for more info)
\usepackage{nameref,hyperref}

% line numbers
\usepackage[right]{lineno}

% ligatures disabled
\usepackage[nopatch=eqnum]{microtype}
\DisableLigatures[f]{encoding = *, family = * }

% color can be used to apply background shading to table cells only
%\usepackage[table]{xcolor}

% array package and thick rules for tables
\usepackage{array}

% create "+" rule type for thick vertical lines
\newcolumntype{+}{!{\vrule width 2pt}}

% create \thickcline for thick horizontal lines of variable length
\newlength\savedwidth
\newcommand\thickcline[1]{%
  \noalign{\global\savedwidth\arrayrulewidth\global\arrayrulewidth 2pt}%
  \cline{#1}%
  \noalign{\vskip\arrayrulewidth}%
  \noalign{\global\arrayrulewidth\savedwidth}%
}

% \thickhline command for thick horizontal lines that span the table
\newcommand\thickhline{\noalign{\global\savedwidth\arrayrulewidth\global\arrayrulewidth 2pt}%
\hline
\noalign{\global\arrayrulewidth\savedwidth}}


% Remove comment for double spacing
%\usepackage{setspace} 
%\doublespacing

% Text layout
\raggedright
\setlength{\parindent}{0.5cm}
\textwidth 5.25in 
\textheight 8.75in

% Bold the 'Fig #' in the caption and separate it from the title/caption with a period
% Captions will be left justified
\usepackage[aboveskip=1pt,labelfont=bf,labelsep=period,justification=raggedright,singlelinecheck=off]{caption}
\renewcommand{\figurename}{Fig}

% Use the PLoS provided BiBTeX style
\bibliographystyle{plos2015}

% Remove brackets from numbering in List of References
\makeatletter
\renewcommand{\@biblabel}[1]{\quad#1.}
\makeatother



% Header and Footer with logo
\usepackage{lastpage,fancyhdr,graphicx}
\usepackage{epstopdf}
\usepackage{lmodern}
%\pagestyle{myheadings}
\pagestyle{fancy}
\fancyhf{}
%\setlength{\headheight}{27.023pt}
%\lhead{\includegraphics[width=2.0in]{PLOS-submission.eps}}
\rfoot{\thepage/\pageref{LastPage}}
\renewcommand{\headrulewidth}{0pt}
\renewcommand{\footrule}{\hrule height 2pt \vspace{2mm}}
\fancyheadoffset[L]{2.25in}
\fancyfootoffset[L]{2.25in}
\lfoot{\today}

%% Include all macros below

\newcommand{\lorem}{{\bf LOREM}}
\newcommand{\ipsum}{{\bf IPSUM}}

%% END MACROS SECTION

%% personal packages and macro
%%% packages
\usepackage[utf8]{inputenc}        % allow utf-8 input
\usepackage[T1]{fontenc}           % use 8-bit T1 fonts
\usepackage[dvipsnames, table]{xcolor}
\usepackage{tabularx}
\usepackage{multirow}
\usepackage{pifont}
\usepackage{csvsimple}
\usepackage[font={small},textfont={it},labelfont={bf}]{caption}
\usepackage{subcaption}
\usepackage{graphicx}
\usepackage{url}                   % simple URL typesetting
\usepackage{booktabs}              % professional-quality tables
\usepackage{makecell}
\usepackage{amsfonts}              % blackboard math symbols
\usepackage{amsmath}
\usepackage{nicefrac}              % compact symbols for 1/2, etc.
\usepackage{microtype}             % microtypography
\usepackage{enumitem}
\usepackage[export]{adjustbox}


%not compatible with cite package
%\usepackage[natbib=true,style=nature,maxnames=999,maxcitenames=2,backend=biber]{biblatex}
%\addbibresource{references.bib}

%%% macros
\DeclareMathOperator*{\argmin}{arg\,min}
\newcommand{\indep}{\perp \!\!\! \perp}
\newtheorem{assumption}{Assumption}

\definecolor{dark_blue}{rgb}{0,0,.65}
\definecolor{dark_green}{rgb}{0,.5,.15}

\hypersetup{pdftex,  % needed for pdflatex
  breaklinks=true,  % so long urls are correctly broken across lines
  colorlinks=true,
  linkcolor=dark_blue,
  citecolor=dark_green,
}
\colorlet{P}{ForestGreen}
\colorlet{I}{MidnightBlue}
\colorlet{C}{YellowOrange}
\colorlet{O}{DarkOrchid}
\colorlet{T}{Gray}



\begin{document}
\vspace*{0.2in}

\section*{Supporting information}

\paragraph*{S1 Appendix}
\label{apd:causal_assumptions}
{\bf Assumptions: what is needed for causal inference from observational studies.}


The following four assumptions, referred as strong ignorability, are
needed to assure identifiability of the causal estimands with observational
data with most causal-inference methods \cite{rubin2005causal}, in
particular these we use:

\begin{assumption}[Unconfoundedness]\label{assumption:ignorability}
    \begin{equation}\label{eq:ignorability}
        \{Y(0), Y(1) \} \indep A | X
    \end{equation}
    This condition --also called ignorability-- is equivalent to the conditional
    independence on the propensity score $e(X)=\mathbb P(A=1|X)$ \cite{rosenbaum1983central}: $\{Y(0), Y(1) \}\indep  A | e(X)$.
\end{assumption}

Unconfoundedness is a strong assumption that might be violated in practice. The
existence of residual bias trough unobserved confounders can be mitigated with
different strategies. The \textit{omitted variable bias} framework encourages
sensitivity analyses allowing to derive bounds on the causal estimate by making
assumptions on the strength of association of the omitted variable with both the
treatment and the outcome. We refer to \cite{cinelli2020making} for a clear
introduction under linear assumption and to \cite{chernozhukov2022long} for an
extension to general non-linear settings. In case of strong unobserved
confounders for which proxy variables can be measured, \textit{proximal
    inference} can be used to obtain identifiability
\cite{tchetgen2024introduction}. These methods require expert knowledge to
classify the proxy between treatment and outcome proxy, after which a two-stage
regression is run to recover the causal effect. Lastly, natural experiments, when
available, should be exploited to estimate causal effects without the need of
unconfoundedness. Instrumental variable methods exploit randomness influencing
the treatment but unrelated to the outcome to simulate a randomized experiment
\cite[chapter 9]{wager2020stats}. Regression discontinuity designs leverage
discontinuous treatment assignment mechanisms with the assumption of a
continuous outcome \cite[chapter 5]{wager2020stats}.

\begin{assumption}[Overlap, also known as Positivity]\label{assumption:overlap}
    \begin{equation}\label{eq:overlap}
        \eta < e(x) < 1 - \eta \quad \forall x \in \mathcal{X} \text{ and some }   \eta > 0
    \end{equation}
    The treatment is not perfectly predictable. Or in other words, every
    patient has a chance to be treated and not to be treated. For a given set of
    covariates, we need examples of both to recover the ATE.
\end{assumption}

As noted by \cite{damour2020overlap}, the choice of covariates $X$ can
be viewed as a trade-off between these two central assumptions. A bigger
covariate set generally reinforces the ignorability assumption. In the
contrary, overlap can be weakened by large $\mathcal{X}$ because of the
potential inclusion of instrumental variables: variables only linked to the treatment which
could lead to arbitrarily small propensity scores.


% remark: There is a major counter example to these colliders (variables which
% are caused by both the outcome and the treatment),

\begin{assumption}[Consistency]\label{assumption:consistency} The observed
    outcome is the potential outcome of the assigned treatment:
    \begin{equation}\label{eq:consistancy}
        Y = A \, Y(1) + (1-A) \, Y(0)
    \end{equation}
    Here, we assume that the intervention $A$ has been well defined. This
    assumption focuses on the design of the experiment. It clearly states the link
    between the observed outcome and the potential outcomes through the
    intervention \cite{hernan2020causal}.
\end{assumption}

\begin{assumption}[Generalization]\label{assumption:generalization} The training
    data on which we build the estimator and the test data on which we make the
    estimation are drawn from the same distribution, also known as
    the ``no covariate shift'' assumption \cite{jesson2020identifying}.
\end{assumption}


\bibliography{references}


\end{document}
