
\section{Former study on brain imaging}


\subsection{Neuro-imaging for stroke patients}
\paragraph{Framing the question}

\ts{Problems I see is you might be mixing two populations: A) people being
  admitted because of a stroke. This stroke is usually severe otherwise the would
  go to dedicated intermediate care stroke units. As such they receive multiple
  CTs because they are intubated and cannot communicate -> only way to assess
  stroke progress is by imaging. B) people being admitted because of something
  else -> either develop symptoms of a stroke (treated and billed correctly) or
  maybe even have a incidental finding of a stroke that's being coded for billing
  reasons.}
\begin{itemize}
  \item P: The population of interest is defined as patient aged over 18, hospitalized
        in ICU with a billing diagnoses of stroke or TIA (ICD9 codes 430-439) as main diagnosis which have
        at least 24 hours of observation. We hypothesize that there is no missing stroke
        billing code, at least no missingness associated to the treatment. This assumption
        is equivalent to say that whereas a patient with stroke has a brain imaging or
        not, it will be coded with TIA/stroke. \md{Is this reasonnable ? Do we
          opportunistically discover a stroke by CT-scan ? Or are the symptoms
          sufficiently clear for the initial diagnosis to be independant.} The 24 hour
        required survival is an attempt to rule out extremely severe patients that would
        have require more urgent care than a brain imaging, for example if the
        circulatory or respiratory is failing at ICU admission.
        At this step, the cohort comprises 2 801 ICU stays with 2 646 distinct patients.
  \item T: We search for brain imaging during the first 24 hours (or 12 first hours ?).
  \item C: No brain imaging during the first 24 hours.
  \item O: Mortality 28-days after ICU-admission. The followup starts after the
        first 24 hours of survival.
\end{itemize}

This gives the following medical question: What is the effect of
\textcolor{I}{performing neuro-imaging evaluation} vs \textcolor{C}{no
  neuro-imaging} for patient admitted in critical care on \textcolor{O}{28-day mortality}

\paragraph{Preprocessing}

\paragraph{Variable selection}

\ts{What do I have to do here as a reseracher ? Suggest drawing a DAG -> write down all possible confounders. }
\begin{itemize}
  \item 13 baseline measures: We are using the same 17 ICU measured clinical variables as in
        \citep{harutyunyan2019multitask}: Capillary refill rate (binary), Diastolic blood
        pressure, Fraction inspired oxygen, Glasgow coma scale eye opening (ordinal, 1
        to 4), Glasgow coma scale motor response(1 to 6), Glasgow coma scale total (1 to
        15), Glasgow coma scale verbal response (1 to 5), Glucose, Heart Rate, Height,
        Mean blood pressure, Oxygen saturation, Respiratory rate, Systolic blood
        pressure, Temperature, Weight, pH. A summary table of the input features on the
        target population is given in Appendix B.4.
  \item Expert set
  \item All available measures \ts{Why do I need all variables or measures ? How do I handle them ? Any benefit ?}
\end{itemize}

\paragraph{Variable aggregation}

As a baseline for treatment effect estimation, we consider summary statistics of
the stays as implemented by \citep{harutyunyan2019multitask}. Their script computes 6
different statistical features for a given time series (min, max, mean, sd, skew
and number of measurements) for 7 subsequences of the input sequence (the full
time series, the first 10, 25, 50\% and the last 50, 25, 10\% of time points).
This preprocessing of raw features leads to a vector of dimension 714 describing
each ICU stay. We normalize and impute with Z-score the resulting data.

\ts{Reason to do this ? What happens if I ignore this ?}

\paragraph{Identification}

\paragraph{Estimation}

\paragraph{Sensitivity analysis}


\begin{figure}
  \centering
  \includegraphics[width=0.8\linewidth]{img/ate_variability_MIMIC-III__variable_baselines.png}
  \caption{The different choices made in the inference pipeline have significant
    impact on the treatment effect results, requiring to question each model
    causal assumptions such as residual confounding or population overlap at the variable representation
    step, model specification at the estimation step.}
\end{figure}
